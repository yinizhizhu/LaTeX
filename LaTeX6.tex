\documentclass[a4paper,11pt,twocolumn]{article}
\begin{document}
\flushleft

\begin{enumerate}
\item You can mix the list
environments to your taste:
\begin{itemize}
\item But it might start to
look silly.
\item[-] With a dash.
\end{itemize}

\item Therefore remember:
\begin{description}
\item[Stupid] things will not
become smart because they are
in a list.
\item[Smart] things, though, can be
presented beautifully in a list.
\end{description}
\end{enumerate}


\begin{flushleft}
This text is\\ left-aligned.
\LaTeX{} is not trying to make
each line the same length.
\end{flushleft}

\begin{flushright}
This text is right-\\aligned.
\LaTeX{} is not trying to make
each line the same length.
\end{flushright}

\begin{center}
At the centre\\of the earth
\end{center}

A typographical rule of thumb
for the line length is:
\begin{quote}
On average, no line should
be longer than 66 characters.
\end{quote}
This is why \LaTeX{} pages have
such large borders by default and
also why multicolumn print is
used in newspapers.

I know only one English poem by
heart. It is about Humpty Dumpty.
\begin{flushleft}
\begin{verse}
Humpty Dumpty sat on a wall:\\
Humpty Dumpty had a great fall.\\
All the King’s horses and all
the King’s men\\
Couldn’t put Humpty together
again.
\end{verse}
\end{flushleft}


The \verb|\ldots| command \ldots
\begin{verbatim}
10 PRINT "HELLO WORLD ";
20 GOTO 10
\end{verbatim}


\begin{verbatim*}
the starred version of
the verbatim
environment emphasizes
the spaces in the text
\end{verbatim*}

\verb*|like this :-) | 

\end{document}